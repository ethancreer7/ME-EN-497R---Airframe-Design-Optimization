\documentclass{article}
\usepackage{amsmath}
\usepackage{graphicx}
\graphicspath{{figures/}}
\usepackage{subcaption}
\usepackage{adjustbox}
\usepackage{float}
\usepackage{xcolor}
\usepackage{array}
\usepackage{gensymb}
\usepackage{caption}
\definecolor{linkblue}{RGB}{6,125,233}
\usepackage{hyperref}
\hypersetup{
    colorlinks=true,
    linkcolor=linkblue,
    urlcolor=linkblue,
    citecolor=linkblue
}
\usepackage{booktabs}
\title{Convergence Study of a Simple Wing}
\author{Ethan Creer}
\date{1 November 2025}
\begin{document}
\maketitle
\begin{abstract}
    A simple convergence study using FlowUnsteady.jl and FlowVLM.jl performed on a simple wing. Primary objective of finding variable values that provides sufficiently converged coefficient of lift and drag values.
    For emphasis and further exploration on the actual convergence process, the tolerance was set particularly low --- a 0.1\% change from previous run.
\end{abstract}
\section{Problem Setup}

Before performing a preliminary convergence study analysis, it is necessary to define the wing geometry and simulation parameters used in the study.
These parameters are defined as found in Table~\ref{tab:geometry}

\begin{table}[H]
    \centering
    \caption{Wing geometry and simulation parameters with their given values.}\label{tab:geometry}
    \begin{tabular}{l l l}
        \toprule
        \textbf{Parameter} & \textbf{Meaning} & \textbf{Value} \\
        \midrule
        $\alpha$ & Angle of Attack & 4.2\degree \\
        $V_{\infty}$ & Freestream Velocity & 49.7 m/s \\
        $\rho$ & Air Density & 0.93 kg/$m^3$ \\
        b & Span Length & 2.489 $m$ \\
        AR & Aspect Ratio b/${c_{tip}}$ & 5.0 \\
        TR & Taper Ratio ${c_{tip}}/{c_{root}}$ & 1.0 \\
        $\lambda$ & Sweep Angle & 45.0\degree \\
        $\gamma$ & Dihedral & 0.0\degree \\

    \end{tabular}
\end{table}




\begin{table}[H]
    \centering
    \caption{A list of adjustable variables that alter the resolution of the output}\label{tab:adjustables}
    \begin{tabular}{c c c}
        \toprule
        \textbf{Variable} & \textbf{Meaning} & \textbf{Convergence Mode} \\
        \midrule
        n & Number of Spanwise Elements per Side & Spatial \\
        nsteps & Number of Time Steps & Temporal \\
        \bottomrule
    \end{tabular}
\end{table}





\end{document}