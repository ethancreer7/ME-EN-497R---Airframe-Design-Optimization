\documentclass{article}
\usepackage{amsmath}
\usepackage{graphicx}
\graphicspath{{figures/}}
\usepackage{subcaption}
\usepackage{adjustbox}
\usepackage{float}
\usepackage{xcolor}
\usepackage{array}
\usepackage{siunitx}
\usepackage{caption}
\definecolor{linkblue}{RGB}{6,125,233}
\usepackage{hyperref}
\hypersetup{
    colorlinks=true,
    linkcolor=linkblue,
    urlcolor=linkblue,
    citecolor=linkblue
}
\usepackage{booktabs}
\title{Convergence Study of a Propeller}
\author{Ethan Creer}
\date{15 November 2025}
\begin{document}


Before performing a preliminary convergence study analysis, it is necessary to define the propeller geometry and simulation parameters used in the study.
These parameters are defined as found in Table~\ref{tab:geometry}

\begin{table}[H]
    \centering
    \caption{Propeller geometry and simulation parameters with their given values. For simplicity, only the overall rotor is defined and reference. The blade geometry is dictated by its model (an APC 10$\times$7 rotor) with relevant geometry found on the BYU Flow Lab \href{https://github.com/byuflowlab/FLOWUnsteady/tree/master/database/rotors}{github}.}\label{tab:geometry}
    \begin{tabular}{l l l}
        \toprule
        \textbf{Parameter} & \textbf{Description} & \textbf{Value} \\
        \midrule
        $R_{tip}$ & Radius of Blade Tip (m) & 0.127 \\
        $R_{hub}$ & Radius of Hub (m) & 0.0095325 \\
        $B$ & Number of Blades & 2 \\
    \end{tabular}
\end{table}


With the wing now defined, we are now set to begin the convergence study. We are primarily interested in the variables listed in Table~\ref{tab:adjustables}.
The convergence mode for each is listed in the same table giving us a preliminary idea of how each variable will affect the convergence of the coefficient of thrust.


\begin{table}[H]
    \centering
    \caption{A list of adjustable variables that alter the resolution of the output coefficient of thrust}\label{tab:adjustables}
    \begin{adjustbox}{center}
    \begin{tabular}{l l c}
        \toprule
        \textbf{Variable} & \textbf{Meaning} & \textbf{Convergence Mode} \\
        \midrule
        nrevs & Number of Propeller Revolutions & Temporal \\
        nsteps\_per\_rev & Number of Time Steps per Propeller Revolution & Temporal \\
        $n$ & Number of Spanwise Elements per Propeller Blade & Spatial \\
        p\_per\_step & Particles shed per step & Spatial \\
        \bottomrule
    \end{tabular}
    \end{adjustbox}
\end{table}

Our convergence study will follow a systematic approach, gradually iterating through variables to suggest final parameters such that the propeller will be appropriately discretized and sufficiently converged.

\section{Number of Revolutions $\left(nrevs\right)$}
The study to determine how many revolutions of the propeller are needed before the value for the coefficient of thrust is sufficiently converged is a relatively simple one.
This study can be done via a visual inspection of a CT vs time plot, like the one shown in Figure~\ref{fig:ct_vs_time_init.png}
What we are looking for in this plot is at what point the coefficient of thrust begins to level off. We can then estimate the number of revolutions $\left(to the nearest whole number\right)$ before the coefficient levels off.
The simulation results depicted in Figure~\ref{fig:ct_vs_time_init} ran for a total of 8 revolutions, and we see that the plot begins to level off about a third of the way through the simulation time.
We thus take a third of 8 revolutions, round to the nearest whole number of the 3 revolutions, and take that to be our converged number of revolutions.
This value will be confirmed again in our verification study of our converged recommended values.

\begin{figure}
    \centering
    \includegraphics[width=0.9\textwidth]{CT_vs_time_nrevs8.png}
    \caption{A plot representing the temporal convergence of the coefficient of thrust. Convergence depicted by the plot leveling off}\label{fig:ct_vs_time_init}
\end{figure}

\begin{figure}
    \centering
    \includegraphics[width=0.9\textwidth]{lofts.png}
    \caption{A 3D representation of the rotor blade geometry}\label{fig:lofts}
\end{figure}

\begin{figure}
    \centering
    \includegraphics[width=0.9\textwidth]{verified_color_adjust_glyphs.png}
    \caption{A paraview depiction of 120,000 randomly selected particles (approximately an eigth of the total particle count) with gamma coloring set based off of points near the rotors}\label{fig:verified_color_adjust_glyphs}
\end{figure}

\begin{figure}
    \centering
    \includegraphics[width=0.9\textwidth]{verified_true_glyphs.png}
    \caption{A paraview depiction of 120,000 randomly selected particles (approximately an eigth of the total particle count) with gamma coloring set based off of points at the tail end of the wake}\label{fig:verified_true_glyphs}
\end{figure}


\begin{figure}
    \centering
    \includegraphics[width=0.9\textwidth]{verified_color_adjust_pfield.png}
    \caption{A surface paraview representation of the full particle field with associated gamma coloring set based on the particles near the rotor}\label{fig:verified_pfield}
\end{figure}



\begin{table}[H]
    \centering
    \caption{A list of recommended parameter values to produce a converged simulation}\label{tab:suggested}
    \begin{adjustbox}{center}
    \begin{tabular}{l c}
        \toprule
        \textbf{Parameter} & \textbf{Value} \\
        \midrule
        nrevs & 3  \\
        nsteps\_per\_rev & 240 \\
        $n$ & 40 \\
        \bottomrule
    \end{tabular}
    \end{adjustbox}
\end{table}



\end{document}


